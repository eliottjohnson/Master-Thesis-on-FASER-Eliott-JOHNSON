% ************************** Thesis Abstract *****************************
% Use `abstract' as an option in the document class to print only the titlepage and the abstract.
\begin{abstract}

ATLAS and CMS, the biggest experiments at the LHC, scrutinize the high transverse momentum regime of proton-proton collisions to discover new physics. However, if new particles are light and weakly coupled, this focus may be completely misguided: light particles are typically highly concentrated within a few mrad of the beam line, out of reach of ATLAS and CMS.
FASER is a small and inexpensive detector to be installed in the far forward region of proton-proton collisions. It will search for weakly interacting light particles produced in the decays of light mesons - mainly $\pi$ and K - that are abundantly produced in this region. It will collect data during Run 3 and probe new regions of the parameter space for new particles with masses in the \SI{10}{\MeV} to GeV range.
This thesis will briefly describe the experiment, followed by a detailed explanation of the data acquisition and the trigger logic board used in FASER.
\end{abstract}
