%!TEX root = ../thesis.tex
%*******************************************************************************
%****************************** Fourth Chapter **********************************
%*******************************************************************************
\chapter{Conclusion}

% **************************** Define Graphics Path **************************
\ifpdf
    \graphicspath{{Conclusion/Figs/Raster/}{Conclusion/Figs/PDF/}{Conclusion/Figs/}}
\else
    \graphicspath{{Conclusion/Figs/Vector/}{Conclusion/Figs/}}
\fi

No new physics in high $\text{p}_{T}$ of $pp$ collisions has led the search for light and weakly coupled LLPs and the design and construction of FASER, a small and inexpensive detector installed in the far forward region of $pp$ collision probing new regions of the parameter space that might confirm the existence of BSM particles such as the axion, suggested in 1978, by F. Wilczek and S. Weinberg \cite{weinberg_new_1978,wilczek_problem_1978}.

Hardware commissioning and peer-reviewed software development of the Trigger Logic Board as well as it's integration in the DAQ framework allow the FASER collaboration to further the development of the experiment by allowing the measurement of cosmic rays using the scintillators and more complex system testing such as tracker modules communicating to the TLB through the TRBs. The TLB is an essential component for FASER's operation during Run 3 and can be modified with ease to function in the future on the bigger and more advanced FASER 2 that will extend the sensitivity and include new particles produced in heavy meson decays.