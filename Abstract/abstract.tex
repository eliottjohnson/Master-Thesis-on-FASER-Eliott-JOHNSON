% ************************** Thesis Abstract *****************************
% Use `abstract' as an option in the document class to print only the titlepage and the abstract.
\begin{abstract}



The Large Hadron Collider is one of the wonders of the modern world - the largest and highest-energy particle collider designed to explore the laws governing the interactions and forces among the fundamental particles, the structure of space and time and the correlation between quantum mechanics and general relativity. It has achieved to study in great detail the Standard Model as well as discover the Higgs Boson, the particle that gives mass to all other particles, in 2012. Many fundamental questions in physics remain open such as the apparent violations of symmetry between matter and antimatter, the Hierarchy problem between the four fundamental forces and the Grand Unification Theories to name a few.
ATLAS and CMS, the biggest detectors of the LHC, scrutinize the high $\text{p}_{T}$ regime of $pp$ collisions to discover new physics. However, if new particles are light and weakly coupled, this focus may be completely misguided: light particles are typically highly concentrated within a few mrad of the beam line, out of reach of ATLAS and CMS.
FASER is a small and inexpensive detector to be installed in the far forward region of $pp$ collision. It will search for weakly interacting light particles produced in the decays of light mesons - mainly $\pi$ and K - that are abundantly ($\sim10^{16}$ $\pi$ per year) produced in this region. It will collect data during Run 3 and probe new regions of the parameter space for new particles with masses in the 10 MeV to GeV range.
This thesis will briefly explain the experiment and then go into more details in the data acquisition and the trigger logic board used in FASER.
\end{abstract}
