%!TEX root = ../thesis.tex
%*******************************************************************************
%****************************** Fourth Chapter **********************************
%*******************************************************************************
\chapter{Conclusion and Outlook}

% **************************** Define Graphics Path **************************
\ifpdf
    \graphicspath{{Conclusion/Figs/Raster/}{Conclusion/Figs/PDF/}{Conclusion/Figs/}}
\else
    \graphicspath{{Conclusion/Figs/Vector/}{Conclusion/Figs/}}
\fi

No presence of new physics in high $\text{p}_{T}$ of $pp$ collisions has led the search for light and weakly coupled particles with a characteristic long lifetime. These will be searched in the new ForwArd Search ExpeRiment (FASER), a small and inexpensive detector installed in the far forward region of $pp$ collision. FASER will probe new regions of the parameter space that might confirm the existence of Beyond the Standard Model particles such as the axion or the dark photon. FASER has a small 5 m long footprint consisting of magnets, tracker planes, scintillators and a small calorimeter. A trigger and data acquisition system (TDAQ) provides the experiment with trigger and read-out capability using a Trigger Logic Board (TLB) as the main component.

Hardware commissioning and peer-reviewed software development of the TLB has been completed in the context of this master project. As a result, the TLB is now ready to be used for commissioning the FASER experiment. Integration tests with other components of the TDAQ system have taken place in a dedicated laboratory at CERN and further testing is in progress. An important milestone will be cosmic data taking  with the TLB  providing triggers to read out detector components upon the passage of cosmic rays; this will initially be done using scintillators and calorimeter modules and soon after, with tracker planes.  

The TLB is an essential component for FASER’s operation during Run 3. It can be modified to function in the future on the prospective bigger and more advanced FASER 2 experiment that will extend the sensitivity to include new particles produced in heavy meson decays. 
